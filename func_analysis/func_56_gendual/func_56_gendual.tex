\documentclass[12pt,fleqn]{article}\usepackage{../../common}
\begin{document}
Genel �ifteslik / �kizlik (General Dual�ty)

Ders notlari [1]'den alinmistir. 

Bu derste genel d��b�key problemlerinde ikizlik konusunu g�rece�iz, ek
olarak d��b�key olmayan durumlara da bakaca��z. �kizlik oldukca genel bir
konu. 

�nceki derste ikizli�i elde etmenin ikinci y�nteminde 

$$
L(x,u,v) \equiv c^T x + u^T (Ax-b) + v^T (Gx-h) 
$$

tan�m� �zerinden (Lagrangian)

$$
f^* = \min_{x \in C} L(x,u,v) \ge \min_x L(x,u,v) \equiv g(u,v)
$$

�eklinde bir form�l elde etmi�tik, $u,v$ sabitlendi�i durumda ve $x \in C$
olurlu ��z�mleri k�mesi olacak �ekilde, ve bu k�me �zerinden Lagrangian'in,
t�m $x$'ler �zerinden olan Lagrangian'dan her zaman daha b�y�k olaca��n�
g�rm��t�k. Asl�nda bu �ok basit bir fikir ama bu basit ve kuvvetli fikir
sayesinde ikizli�i genel problemlere uygulamak m�mk�n oluyor. 






[devam edecek]

Kaynaklar

[1] Tibshirani, {\em Convex Optimization, Lecture Video 11}, 
\url{https://www.youtube.com/channel/UCIvaLZcfz3ikJ1cD-zMpIXg}   

\end{document}




















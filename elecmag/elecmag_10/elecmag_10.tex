\documentclass[12pt,fleqn]{article}\usepackage{../../common}
\begin{document}
Ders 10

�nceki derste

\begin{itemize}
   \item Elektriksel potansiyel (sonsuz ay�r�ma g�re olan voltaj)
   \item Potansiyel fark ve elektrik alan�
   \item Potansiyel fark hesab�n�n yoldan ba��ms�z olmas�
\end{itemize}

Bug�n i�lenecek konular

\begin{itemize}
   \item Elektrik potansiyel �zerinden ge�mek
   \item Tek noktadaki potansiyel
   \item Bir iletken i�indeki potansiyel
   \item Bir yal�tkan i�indeki potansiyel
   \item Bir alanda depolanan enerji
\end{itemize}

Tek noktadaki elektrik potansiyelle ba�layal�m: bu hesap m�mk�n m�? Elektrik
potansiyel fark�n� $E$'nin bir mesafe �zerinen entegralinin negatifi olarak
g�rm��t�k, bir �izgi entegraliydi.

$$
\Delta V \equiv -\int _{a}^{b} \vec{E} \cdot \ud \vec{x}
$$







\end{document}

\documentclass[12pt,fleqn]{article}\usepackage{../../common}
\begin{document}
Konusma Tanima (Speech Recognition)

CTC

\inputminted[fontsize=\footnotesize]{python}{train_vctk.py}

Frekans Uzerinden Ozellik Cikartimi, RNN, LSTM, GRU

\inputminted[fontsize=\footnotesize]{python}{speech1.py}


[devam edecek]



Kaynaklar

[1] Bayramli, {\em VCTK Ses Tanima Verisi, Konusmaci 225}, \url{https://www.dropbox.com/s/xecprghgwbbuk3m/vctk-pc225.tar.gz?dl=1}

[2] Remy, {\em Application of Connectionist Temporal Classification (CTC) for Speech Recognition},\url{https://github.com/philipperemy/tensorflow-ctc-speech-recognition}

[3] Graves, {\em Supervised Sequence Labelling with Recurrent Neural Networks}, \url{https://www.cs.toronto.edu/~graves/preprint.pdf}

[4] Graves, {\em How to build a recognition system (Part 2): CTC Loss}, \url{https://docs.google.com/presentation/d/12gYcPft9_4cxk2AD6Z6ZlJNa3wvZCW1ms31nhq51vMk}

[5] Graves, {\em How to build a recognition system (Part 1): CTC Loss}, \url{https://docs.google.com/presentation/d/1AyLOecmW1k9cIbfexOT3dwoUU-Uu5UqlJZ0w3cxilkI}

[6] Bayramli, {\em Small Voice Files}, \url{https://www.dropbox.com/s/8754ytatft2ttge/voice_cmd_small.zip?dl=1}

\end{document}

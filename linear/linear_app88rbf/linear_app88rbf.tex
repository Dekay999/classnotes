\documentclass[12pt,fleqn]{article}\usepackage{../../common}
\begin{document}
Dairesel Baz Fonksiyonlari (Radial Basis Functions)

\begin{minted}[fontsize=\footnotesize]{python}
from mpl_toolkits.mplot3d import Axes3D
import matplotlib.pyplot as plt
from matplotlib import cm
import numpy as np
import matplotlib.pyplot as plt

def func(x, y):
    s1 = 0.2; x1 = 36.5; y1 = 32.5
    g1 = np.exp( -4 *np.log(2) * ((x-x1)**2+(y-y1)**2) / s1**2)
    s2 = 0.4; x2 = 36.1; y2 = 32.8
    g1 = np.exp( -4 *np.log(2) * ((x-x1)**2+(y-y1)**2) / s1**2)
    g2 = np.exp( -2 *np.log(2) * ((x-x2)**2+(y-y2)**2) / s2**2)
    return g1 + g2

D = 200
x = np.linspace(36,37,D)
y = np.linspace(32,33,D)

xx,yy = np.meshgrid(x,y)
zz = func(xx,yy)

xxx = xx.reshape(D*D)
yyy = yy.reshape(D*D)
zzz = zz.reshape(D*D)
idx = np.random.choice(range(D*D),500)

xr = xxx[idx]
yr = yyy[idx]
zr = zzz[idx]
print (xr.shape)
\end{minted}

\begin{verbatim}
(500,)
\end{verbatim}

\begin{minted}[fontsize=\footnotesize]{python}
fig = plt.figure()
ax = fig.gca(projection='3d')
surf = ax.plot_surface(xx, yy, zz, cmap=cm.coolwarm,linewidth=0, antialiased=False)
fig.colorbar(surf, shrink=0.5, aspect=5)
plt.savefig('/data/data/com.termux/files/home/Downloads/out1.png')
\end{minted}

\begin{minted}[fontsize=\footnotesize]{python}
gamma = 0.2
xxx = np.subtract.outer(xr,xr).T
yyy = np.subtract.outer(yr,yr).T
aaa = np.exp(-gamma * np.sqrt(xxx**2 + yyy**2))
print (aaa.shape)
import scipy.linalg as lin
w = lin.solve(aaa, zr)
print (w[:5])

xm = [36.4,32.4]
tmp = np.exp(-gamma * np.sqrt((xm[0]-xr)**2 + (xm[1]-yr)**2))
res = np.dot(w,tmp)
print (res)
\end{minted}

\begin{verbatim}
(500, 500)
[-0.19300543 -0.62116301  0.00370977  0.09326682  0.05787695]
0.37171564421857206
\end{verbatim}


\begin{minted}[fontsize=\footnotesize]{python}
rmse = 0
for i in range(len(xr)):
    tmp = np.exp(-gamma * np.sqrt((xr[i]-xr)**2 + (yr[i]-yr)**2))
    rmse += ((np.dot(w,tmp)-zr[i])**2)
rmse = np.sqrt(np.mean(rmse))
print (rmse)
\end{minted}

\begin{verbatim}
3.123703758799e-13
\end{verbatim}





\begin{minted}[fontsize=\footnotesize]{python}
gamma = 0.1
aa = np.array([[1,4],[2,6],[3,8]])
ax = np.subtract.outer(aa[:,0],aa[:,0]).T
bx = np.subtract.outer(aa[:,1],aa[:,1]).T
print (ax.T)
print (bx.T)
cc = np.exp(gamma * (ax**2 + bx**2))
print (cc)
\end{minted}

\begin{verbatim}
[[ 0 -1 -2]
 [ 1  0 -1]
 [ 2  1  0]]
[[ 0 -2 -4]
 [ 2  0 -2]
 [ 4  2  0]]
[[1.         1.64872127 7.3890561 ]
 [1.64872127 1.         1.64872127]
 [7.3890561  1.64872127 1.        ]]
\end{verbatim}














Kaynaklar

[1] Neto, {\em Radial Basis Functions}, \url{http://www.di.fc.ul.pt/~jpn/r/rbf/rbf.html}

\end{document}







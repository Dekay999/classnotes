\documentclass[12pt,fleqn]{article}\usepackage{../../common}
\begin{document}
Ders 20

[dersin ba��nda hocan�n yapt��� demo atland�, periyot katlanmas�, lojistik
e�lemeyi g�steren programlar sundu]

Kaosun evrensel mekanizmas�na geldik. Bu k�s�m t�m dersteki en zor
k�s�mlardan biri olacak, o y�zden zihnimizi a��k tutal�m, performans�
artt�racak bir d��me varsa ona basal�m :) �ok abartmay�m tabii, sadece bu
ve takip eden birka� dersin soyutluk ve zorluk seviyesi di�er b�l�mlere
nazaran biraz daha yukar�da. Ama bu zorluk �ekmeye bence de�er ��nk�
konumuzun entellekt�el kazanc� a��s�ndan vard��� en y�ksek nokta
buras�. 

Anlatacaklarim kitabimin 10.6, 10.7 bolumunden, Feigenbaum adli bilimcinin
arastirmasini merkez aliyoruz. 



















Kaynaklar

[1] Feigenbaum, {\em Quantitative universality for a class of nonlinear transformations}, \url{https://www.researchgate.net/publication/226628603_Quantitative_universality_for_a_class_of_nonlinear_transformations}

\end{document}

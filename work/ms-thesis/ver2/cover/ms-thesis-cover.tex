\chapter{Introduction}

FINANCIAL TIME SERIES PREDICTION USING KALMAN FILTERS AND HIDDEN MARKOV MODELS 

Burak Bayramli

B.E., Computer Engineering, Stevens Institute of Technology, 1996

Systems and Control Engineering

Prof. Fikret Gurgen

Prof. Levent Akin

Prof. Ethem Alpaydin

I would like to thank my thesis advisor Dr. Gurgen for his help, direction and
patience during my work on this thesis. 

Thanks to Dr. Ethem Alpaydin for first suggesting Systems and Control
Engineering department as a good master's goal, and second for all the knowledge
I gained in his {\em Pattern Recognition} class. Once in Systems and Control
Engineering department, I was able to take immensely interesting classes such as
{\em Nonlinear Dynamics} and {\em Simulation} both of which helped greatly in
researching this thesis. I am also grateful to Dr. Goldsman for the all
knowledge he was willing to part with in his {\em Simulation} class - I think
his lectures on pseudorandom generation will stay with me forever. It was also
in his class I first saw Brownian Motion which is very important in financial
time series analysis.

I extend warmest thanks to Dr. Levent Akin for his excellent {\em Autonomous
Robotics} class and his suggestion that I look into {\em Probabilistic
Robotics}, a book and term popularized by Sebastian Thrun - it was while
flipping through the pages of this book that I got the idea of using Kalman
Filters for stock price prediction.

I would also like to thank my parents Ercan and Fusun Bayramli for their support
without whom none of this would have been possible.

\newpage

Financial markets are challenging targets for analysis and prediction. The
existence of many factors that contribute to a final price of security at time
$t$, makes the task of predicting a future price a very hard task indeed. In the
past, many statistical and non-statistical models have been utilized that
attempted to perform this daunting task - the aim of this thesis is going to be
trying to demonstrate Hidden Markov Models and Kalman Filter methods can be used
for predicting a future price. We also devised a mixture predictor using HMM and
KF which we called KMM.

For this purpose, we hypothized HMM, KF and KMM based models that are trained on
historical data can generate future data, thereby predicting this securities'
price in the future. For data generation, we used Monte Carlo simulation to
smooth over the irregular patterns. ``Carrying the model forward in time'' is
achieved by a combination of Viterbi algorithm and rolling the dice on hidden
state transitions, in KF case, we follow the time transition equation.

Another goal of this thesis was to determine how HMM, KF and KMM methods stood
in comparison to other conventional methods. We picked polynomial regression,
Neural Networks (ANN) and plain ``random guess'' as our comparison
criteria. Random guess was expected to be the lowest performer in our tests,
every method mentioned should have surpassed random guess results by wide
margin. We were glad to see that this was indeed the case.

\newpage

\tableofcontents
\listoffigures
\listoftables




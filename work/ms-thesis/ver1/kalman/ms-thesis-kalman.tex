%%%%%%%%%%%%%%%%%%%%%%%%%%%%%%%%%%%%%%%%%%%%%%%%%%%%%%%%%%%%%%%%%%%%%%%%%%%%%
%%%%%%%%%%%%%%%%%%%%%%%%%%%%%%%%%%%%%%%%%%%%%%%%%%%%%%%%%%%%%%%%%%%%%%%%%%%%%
\chapter{KALMAN FILTERS}

Compiling an exhaustive list of fields where Kalman Filters (KF) are in use
would be a formiddable task: The list could start with navigation, radar
tracking, satellite orbit computation \cite{welling}, continue with sonar
ranging, stock price prediction (which hopefully this thesis will prove), and
many other fields where there is a dynamical system present. When the {\em
  Eagle} module of the Apollo 11 mission landed on the moon, it is widely known
that it did so with the help of a Kalman Filter. KF was first proposed in a
seminal paper by \cite{kalman}, later Kalman and Bucy \cite{kalman2} - the
method's first application setting was space tracking.

Kalman Filters, or in general, SSM methods try to model dynamic data using
two-level graphic model. In the first level, there are hidden states, one per
time step, which are random variables distributed as Gaussians. An outside
observer (modeler) does not see these states directly, instead sees them through
an ``observed'' variable whose outputs are assumed to be distorted by another
Gaussian distribution, refered to as white noise. In each time step, it is
assumed the hidden variables evolve linearly, in linear algebra this means the
hidden state is multiplied by a constant matrix $A$. Mathematically,
\begin{eqnarray*}
x_{t+1} &=& Ax_t + v_t 
\end{eqnarray*}
where
\begin{itemize}
   \item $v_0,v_1,...$ is white Gaussian noise with $\mathbf{E}v_t = 0$ and $cov
   \: v_t = Q$
   \item $x_t \sim N(\mu_t, \Sigma_t)$.
\end{itemize}
In order to model the output variables, we multiply hidden states with another
constant matrix and add another Gaussian white noise. All together our SSM looks
as follows:
\begin{eqnarray}
x_{t+1} = Ax_t + v_t \label{m1}\\
y_t = Cx_t + w_t \label{m2}
\end{eqnarray}
where $w_0,w_1,...$ are white Gaussian noise with $\mathbf{E}w_t = 0$ and $cov
\:\: w_t = R$. Random variables $w_t$ and $v_t$ are independent of $x_t$
and $y_t$, and from eachother, this assumption is put forth for simplicity. The
dimensions of each variable are:

\begin{itemize}
   \item $x(t) \in \mathbb{R}^n$ is the state
   \item $y(t) \in \mathbb{R}^p$ is the observed output
   \item $v(t) \in \mathbb{R}^n$ is process noise
   \item $w(t) \in \mathbb{R}^p$ is measurement noise
\end{itemize}

Kalman Filters carry the same conditional independence properties that
characterized HMMs - given a state for time $t$, the future is conditionally
independent from the past, hence KF also carries the first-order Markovian
property just like HMMs. Structurally there are many similarities between two
models as well and its methods for inference, as we shall soon see, the main
difference being that HMM uses multinomial distribution for hidden states
whereas KF uses Gaussians.

\begin{figure}[!hbp]
\center{
  \scalebox{0.6}{
  \includegraphics{./kalman/kf.JPG}
  }
}
\caption{Hidden and observed states of a KF}
\end{figure}

Inference with Kalman Filters means estimating the posterior probability given
an observed output sequence. There are two tasks here, filtering and
smoothing. Filtering is formulized by $p(x_t|y_0,...,y_t)$ which tries to
calculate posterior probability for hidden states using measurements up to time
$t$. In smoothing, we will calculate the same probability using measurements
taken after time $t$, which gives us $p(x_t|y_{t+1},...,y_T)$. We also hope to
formulize recursive formulas for both filtering and smoothing.

Both filtering and smoothing have direct consequences in terms of real-life
applications. KFs' natural roots are embedded in space tracking, the method
started as an ``online'' algorithm trying to track down targets using noisy
measurements, as each new measurement is received. In this sense, KF can be seen
as an estimator trying to guess a ``hidden'' position at time $t$ using
measurements up to time $t$. Smoothing, on the other hand, can be thought of a
global ``corrector'' - once we have better knowledge on hand, through smoothing
we can go back to old estimations and correct previous online estimates using
backward recursion.


%%%%%%%%%%%%%%%%%%%%%%%%%%%%%%%%%%%%%%%%%%%%%%%%%%%%%%%%%%%%%%%%%%%%%%%%%%%%%
%%%%%%%%%%%%%%%%%%%%%%%%%%%%%%%%%%%%%%%%%%%%%%%%%%%%%%%%%%%%%%%%%%%%%%%%%%%%%
\section{Filtering}

Following the notation and the derivation steps used by Bishop and Jordan
\cite{jordan}, let's first denote

\begin{itemize}
   \item $\hat{x}_{t}^t \triangleq \mathbf{E}[x_t|y_0,...,y_t]$
   \item $P_{t}^t \triangleq \mathbf{E}[(x_t - \hat{x}_{t|t})(x_t -
   \hat{x}_{t|t})'|y_0,...,y_t]$ 
\end{itemize}
Our first goal is going to be trying to calculate $\hat{x}_{t+1}^{t+1}$ and
$P_{t+1}^{t+1}$ using a new measurement $y_{t+1}$. For this, we need to
``reverse the arrow'' so to speak, in other words formulize the distribution
where $x_t$ has conditional dependence on $y_t$ instead of other way around,
which is what the base equations state right now.

If we take expectation of both sides of \eqref{m1}
\begin{eqnarray*}
\mathbf{E}x_{t+1} = \hat{x}_{t+1} = A\mu_t = A\hat{x}_t
\end{eqnarray*}
Taking covariance of both sides of \eqref{m1} and denoting $P_t$ as $cov \: x(t)$
\begin{eqnarray*}
P_{t+1} = AP_{t}A' + Q
\end{eqnarray*}
This transition is called a ``time update''. It will allow us to propagate the
normal distribution at time $t$ to time $t_1$. The formulas that contain given
$y$'s use the same technique;
\begin{eqnarray*}
\hat{x}_{t+1}^t = Ax_{t}^t\\
P_{t+1}^t = AP_{t}^tA' + Q
\end{eqnarray*}
\begin{eqnarray*}
y_{t+1} &=& Cx_{t+1} + w_t\\
\mathbf{E}[y_{t+1}|y_0,...,y_t] &=& \mathbf{E}[Cx_{t+1}+w_t|y_0,...,y_t]\\
\hat{y}_{t+1}^t &=& C\hat{x}_{t+1}
\end{eqnarray*}
Similarly for covariance
\begin{eqnarray*}
E[(y_{t+1}-\hat{y}_{t+1}^t)(y_{t+1}-\hat{y}_{t+1}^t)'|y_0,...,y_t] =
CP_{t+1}^tC' + R
\end{eqnarray*}

Now the harder task of reversing the arrow: If our goal is formulizing
$p(x_t|y_t)$, then we need to derive the joint distribution between both
variables. Addition of Gaussians is another Gaussian, we can surmize the joint
probability $p(x_t,y_t)$ will be one huge Gaussian.

In order to define the joint normal distribution of $x_{t}$ and $y_{t}$, we need
to write the joint mean and covariance of this big Gaussian. Writing down the
mean will be easy, covariance will be little harder to derive. We can use a
trick \cite{lall} where we write $y_{t} = Cx_{t} + w_t$ as $z = Hu$
where
\begin{eqnarray*}
z =   \left[ \begin{array}{c}
      x_{t} \\
      y_{t}
      \end{array} \right],
H =   \left[ \begin{array}{cc}
      I & 0 \\
      C & I
      \end{array} \right]
u =   \left[ \begin{array}{c}
      x_{t} \\
      w_t
      \end{array} \right],      
\end{eqnarray*}
Now we need to take the covariance of the simpler equation:
\begin{eqnarray*}
cov(z) &=& H cov(u) H'\\
cov(u) &=& \left[ \begin{array}{cc}
      P_{t} & 0 \\
      0 & R
      \end{array} \right]     
\end{eqnarray*}
The full product is:
\begin{eqnarray*}
\left[ \begin{array}{cc}
      I & 0 \\
      C & I
      \end{array} \right]
\left[ \begin{array}{cc}
      P_{t} & 0 \\
      0 & R
      \end{array} \right]
\left[ \begin{array}{cc}
      I & C' \\
      0 & I
      \end{array} \right]           
\end{eqnarray*}
which results in
\begin{eqnarray*}
\left[ \begin{array}{cc}
      P_{t} & P_{t}C' \\
      CP_{t} & CP_{t}C'+R
      \end{array} \right]           
\end{eqnarray*}
We can also re-write this for conditional and with the mean;
\begin{eqnarray}
\left[ \begin{array}{c}
      \hat{x}_{t}^t \\
      C\hat{x}_{t}^t
      \end{array} \right]
\:and\:      
\left[ \begin{array}{cc}
      P_{t}^t & P_{t}^tC' \\
      CP_{t}^t & CP_{t}^tC'+R
      \end{array} \right] \label{jointxy}
\end{eqnarray}
Same for the joint distribution for $x_{t+1}, y_{t+1}$.
\begin{eqnarray}
\left[ \begin{array}{c}
      \hat{x}_{t+1}^t \\
      C\hat{x}_{t+1}^t
      \end{array} \right]
\:and\:      
\left[ \begin{array}{cc}
      P_{t+1}^t & P_{t+1}^tC' \\
      CP_{t+1}^t & CP_{t+1}^tC'+R
      \end{array} \right] \label{jointxy2}
\end{eqnarray}

Now in order to get statements for mean and variance for $x_{t+1}^{t+1}$, we
need to understand partitioned Gaussians. This will guide us when we use
portions of \eqref{jointxy2} and deriving the final equation \cite{jordan}. 

An $n$ dimensional Gaussian distribution can be partitioned into $p$ and $q$
dimensional sub-distributions where $n = p + q$. Hence we can say,

\begin{eqnarray}
\mu = 
\left[ \begin{array}{c}
      \mu_1 \\
      \mu_2
      \end{array} \right]           
\Sigma = 
\left[ \begin{array}{cc}
      \Sigma_{11} & \Sigma_{12}\\
      \Sigma_{21} & \Sigma_{22}
      \end{array} \right] \label{condgeneric}
\end{eqnarray}
\begin{eqnarray*}
p(x|\mu,\Sigma) &=& \frac{1}{(2\pi)^{(p+q)/2}|\Sigma|^{1/2}} \\
&& exp\bigg\{
-\frac{1}{2}
\left( \begin{array}{c}
      x_1 - \mu_1 \\
      x_2 - \mu_2
      \end{array} \right)'
\bigg\}
\left[ \begin{array}{cc}
      \Sigma_{11} & \Sigma_{12}\\
      \Sigma_{21} & \Sigma_{22}
      \end{array} \right]^{-1}
\left( \begin{array}{c}
      x_1 - \mu_1 \\
      x_2 - \mu_2
      \end{array} \right)
\bigg\} 
\end{eqnarray*}
After much algebra, we can obtain equation for $p(x_1|x_2)$. From this, we can
get {\em conditioned} $\mu$ and $\Sigma$, as
\begin{eqnarray}
\mu_{1|2} &=& \mu_1 + \Sigma_{12}\Sigma_{22}^{-1}(x_2 - \mu_2) \label{condx} \\
\Sigma_{1|2} &=& \Sigma_{11}-\Sigma_{12}\Sigma_{22}^{-1}\Sigma_{21} \nonumber
\end{eqnarray}
Now plugging elements from  \eqref{condx} into \eqref{jointxy2} according
to placement in \eqref{condgeneric} we can formulize $\hat{x}_{t+1}^{t+1}$ and
$P_{t+1}^{t+1}$. 
\begin{eqnarray*}
\hat{x}_{t+1}^{t+1} &=& \hat{x}_{t+1}^t + P_{t+1}^tC'
                  (CP_{t+1}^tC'+ \Sigma_{w})^{-1} \\
               &&   (y_{t+1} - C\hat{x}_{t+1}^t) \\
P_{t+1}^{t+1} &=& P_{t+1}^t - P_{t+1}^tC' (CP_{t+1}^tC' + R)^{-1}CP_{t+1}^t
\end{eqnarray*}
If we declare $K_t \triangleq P_{t+1}^tC' (CP_{t+1}^tC'+ \Sigma_{w})^{-1}$ then,
\begin{eqnarray}
\hat{x}_{t+1}^{t+1} &=& \hat{x}_{t+1}^t + K_t (y_{t+1} -
C\hat{x}_{t+1}^t) \label{xtplus1} \\
P_{t+1}^{t+1} &=& P_{t+1}^t - K_t C P_{t+1}^t
\end{eqnarray}

%%%%%%%%%%%%%%%%%%%%%%%%%%%%%%%%%%%%%%%%%%%%%%%%%%%%%%%%%%%%%%%%%%%%%%%%%%%%%
%%%%%%%%%%%%%%%%%%%%%%%%%%%%%%%%%%%%%%%%%%%%%%%%%%%%%%%%%%%%%%%%%%%%%%%%%%%%%
\section{Smoothing}

Filtering involves applying a new measurement $y_{t+1}$ to calculate
$\hat{x}_{t+1}^{t+1}$ and $P_{t+1}^{t+1}$ which is an excellent tool for an
online algorithm. However for stock price prediction, we will have all data we
need already at hand. As opposed to a tracking application where data cannot be
known in advance, we will have our data available, as measurements
$y_1,...,y_T$, in the form of stock prices, to train a Kalman filter based
solution, all at once. Hence we need an equation for $x_{t}'$. For this we need
RTS (Rauch-Tung-Striebel) Smoother.

Based on \eqref{jointxy}, we can write the covariance of joint distribution
$(x_t,x_{t+1})$. In \eqref{jointxy} the multiplication was with $C$, here it's
$A$. Hence,
\begin{eqnarray*}
\left[ \begin{array}{cc}
      P_{t}^t & P_{t}^tA' \\
      AP_{t}^t & AP_{t}^tA'+R
      \end{array} \right]
\end{eqnarray*}
Combining the previous equations together, we need to reach to a statement for
$x_{t}'$. We can start with $\mathbf{E}[x_t|x_{t+1},y_0,...,y_t)]$.
\begin{eqnarray*}
\mu_{1|2} &=& \mu_1 + \Sigma_{12}\Sigma_{22}^{-1}(x_2 - \mu_2)
\end{eqnarray*}
which becomes
\begin{eqnarray*}
\mathbf{E}[x_t|x_{t+1},y_0,...,y_t)] = \hat{x}_{t}^t+P_{t}^tA'(P_{t+1}^t)^{-1}(x_{t+1}-\hat{x}_{t+1}^t)
\end{eqnarray*}
Define $L_t \triangleq P_{t}^tA'(P_{t+1}^t)^{-1}$. Then,
\begin{eqnarray*}
\mathbf{E}[x_t|x_{t+1},y_0,...,y_t)] =
\hat{x}_{t}^t+L_t(x_{t+1}-\hat{x}_{t+1}^t)
\end{eqnarray*}
How about $Var[x_t|x_{t+1},y_0,...,y_t]$? If we remember the joint covariance
for $(x_t,x_{t+1})$
\begin{eqnarray*}
 \left[ \begin{array}{cc} 
      P_{t}^t & P_{t}^tA' \\
      AP_{t}^t & AP_{t}^tA'+R
      \end{array} \right] && \\
\end{eqnarray*}
Then
\begin{eqnarray*}
\Sigma_{1|2} &=& \Sigma_{11}-\Sigma_{12}\Sigma_{22}^{-1}\Sigma_{21}
\end{eqnarray*}
becomes
\begin{eqnarray*}
Var[x_t|x_{t+1},y_0,...,y_t] &=& P_{t}^t - P_{t}^tA'(P_{t+1}^t)^{-1}AP_{t}^t
\end{eqnarray*}
Why?
\begin{eqnarray*}
\left[ \begin{array}{cc}
      P_{t}^t & P_{t}^tA' \\
      AP_{t}^t & AP_{t}^tA'+R
      \end{array} \right]
=      
\left[ \begin{array}{cc}
      P_{t}^t & P_{t}^tA' \\
      AP_{t}^t & P_{t+1}^t
      \end{array} \right]
\end{eqnarray*}
If we try to substitute $L_t \triangleq P_{t}^tA'(P_{t+1}^t)^{-1}$ in to the
equation:
\begin{eqnarray*}
Var[x_t|x_{t+1},y_0,...,y_t] &=& P_{t}^t - P_{t}^t A'(P_{t+1}^t)^{-1}AP_{t}^t \\
&=& P_{t}^t - P_{t}^tA'(P_{t+1}^t)^{-1} (P_{t+1}^t(P_{t+1}^t)^{-1}) AP_{t}^t\\
&=& P_{t}^t - (P_{t}^tA'(P_{t+1}^t)^{-1}) P_{t+1}^t ((P_{t+1}^t)^{-1} AP_{t}^t)\\
&=& P_{t}^t - L_t  P_{t+1}^t L_t'
\end{eqnarray*}
Due to conditional independence, when we condition on $x_{t+1}$, we make $x_t$
independent of future observations $y_{t+1},...,y_T$. Hence we can simply say;
\begin{eqnarray*}
E[x_t|x_{t+1},y_0,...,y_T]  &=& E[x_t|x_{t+1},y_0,...,y_t]\\
&=& \hat{x}_{t}^t +L_t(x_{t+1}-\hat{x}_{t+1}^t) \\
Var[x_t|x_{t+1},y_0,...,y_T]  &=& Var[x_t|x_{t+1},y_0,...,y_t] \\
&=& P_{t}^t - L_t  P_{t+1}^t L_t'
\end{eqnarray*}
We are almost done. If we can drop the $x_{t+1}$ from LHS of both equations, we
can have our smoothing equations. Due to the so-called Tower Property:
\begin{eqnarray*}
\mathbf{E}[X|Z] &=& \mathbf{E}[\mathbf{E}[X|Y,Z]|Z]\\
Var[X|Z] &=& Var[\mathbf{E}[X|Y,Z]|Z] + \mathbf{E}[Var[X|Y,Z]|Z]
\end{eqnarray*}
we can drop $x_{t+1}$ from LHS.
\begin{eqnarray}
\hat{x}_{t}' & \triangleq & \mathbf{E}[x_t|y_0,...,y_T] \nonumber\\
& = & \mathbf{E}[\mathbf{E}[x_t|x_{t+1},y_0,...,y_T]|y_0,...,y_T] \nonumber\\
& = & \mathbf{E}[\hat{x}_{t}^t +L_t(x_{t+1}-\hat{x}_{t+1}^t)] \nonumber\\
& = & \hat{x}_{t}^t +L_t(\hat{x}_{t+1}' -\hat{x}_{t+1}^t) \label{xsmooth}
\end{eqnarray}
Using similar algebra, we get
\begin{eqnarray*}
P_{t}'  = P_{t}^t + L_t(P_{t+1}'-P_{t+1}^t)L_t'
\end{eqnarray*}


%%%%%%%%%%%%%%%%%%%%%%%%%%%%%%%%%%%%%%%%%%%%%%%%%%%%%%%%%%%%%%%%%%%%%%%%%%%%%
%%%%%%%%%%%%%%%%%%%%%%%%%%%%%%%%%%%%%%%%%%%%%%%%%%%%%%%%%%%%%%%%%%%%%%%%%%%%%
\section{Training}

In order to train a Kalman Filter based predictor, we need to estimate
parameters $A,C,Q,R,\pi$. These parameters define a KF.

Making use of Markov property, we write our joint probability as;

\begin{eqnarray*}
p(x_0,..,x_t,y_0,..,y_t) =
p(x_0)\prod_{t=1}^{T-1}p(x_{t+1}|x_t)\prod_{t=0}^{T-1}p(y_t|x_t)
\end{eqnarray*}

\begin{eqnarray*}
p(y_{t}|x_t) &=& exp
\bigg\{
-\frac{1}{2}(y_{t}-Cx_t) ' R^{-1}(y_{t}-Cx_t)
\bigg\}
(2\pi)^{-\frac{p}{2}}|R|^{-\frac{1}{2}}\\
p(x_{t+1}|x_t) &=& exp
\bigg\{
-\frac{1}{2}(x_{t+1}-Ax_t) ' Q^{-1}(x_{t+1}-Ax_t)
\bigg\} \\
&& (2\pi)^{-\frac{k}{2}}|Q|^{-\frac{1}{2}}\\
p(x_{1}) &=& exp
\bigg\{
-\frac{1}{2}(x_{1}-\pi) ' \Sigma_{\pi}^{-1}(x_{1}-\pi)
\bigg\}
(2\pi)^{-\frac{k}{2}}|\Sigma_{\pi}|^{-\frac{1}{2}}
\end{eqnarray*}
We take log
\begin{eqnarray*}
\log p(x,y) &=& \\
&& -\sum_{t=0}^{T-1}(-\frac{1}{2}(y_{t}-Cx_t) ' R^{-1}(y_{t}-Cx_t)) - \frac{T}{2}\log|R|\\
&& -\sum_{t=1}^{T-1}(-\frac{1}{2}(x_{t+1}-Ax_t) ' Q^{-1}(x_{t+1}-Ax_t))\\
&& - \frac{T-1}{2}\log|Q|\\
&& -\frac{1}{2}(x_{1}-\pi) ' \Sigma_{\pi}^{-1}(x_{1}-\pi) - \frac{T(p+k)}{2}\log|\Sigma_{\pi}|
\end{eqnarray*}

In order to estimate parameters, we take derivative of expected log likelihood,
set to zero and solve

\begin{eqnarray*}
Q &=& \mathbf{E}[\log p(x,y|y)] \\
\frac{\partial Q}{\partial C} &=& -\sum_{t=1}^{T}R^{-1}y_t \hat{x}_{t}' +
\sum_{t=1}' R^{-1} C P_t = 0
\end{eqnarray*}

\begin{eqnarray*}
C^{new} &=& \bigg( \sum_{t=1}' y_t \hat{x}_t' \bigg) \bigg( P_{t}' \bigg) ^{-1}
\end{eqnarray*}

Using similar technique, we can calculate other constants as;
\begin{eqnarray*}
R^{new} &=& \frac{1}{T}\sum_{t=1}^T \big( y_ty_t' - C^{new}\hat{x}_ty_t' \big) \\
A^{new} &=& \big( \sum_{t=2}^T P_{t,t-1})(\sum_{t=2}^TP_{t-1} \big) ^{-1} \\
Q^{new} &=& \frac{1}{T-1} \big( \sum_{t=2}^T P_t - A^{new}\sum_{t=2}^TP_{t-1,t} \big)\\
\pi_1^{new} &=& \hat{x}_1
\end{eqnarray*}

%%%%%%%%%%%%%%%%%%%%%%%%%%%%%%%%%%%%%%%%%%%%%%%%%%%%%%%%%%%%%%%%%%%%%%%%%%%
\section{Usage}

While using Kalman Filters for analysis and prediction purposes, one needs to be
careful about starting condition of the training algorithm. Because of the
nature of the training algorithm, the parameters $\mu_i, \sigma_i,A,C,Q,R,P$ are
all used for likelihood calculation/estimation and are {\em themselves} targets
for estimation. The ``chicken and egg'' nature of the EM training process
required us to start somewhere, and that somewhere is picking appropiate
parameters for the parameters mentioned above.

We had to determine said values by trial and error, as part of the process where
we were trying to determine meta parameters such as number of iterations
required for MC process, the lowest likelihood value threshold (which was used
to determine when to stop in the training process - if likelihood difference
fell below this threshold value, we would conclude that training was complete).

%%%%%%%%%%%%%%%%%%%%%%%%%%%%%%%%%%%%%%%%%%%%%%%%%%%%%%%%%%%%%%%%%%%%%%%%%%%
%%%%%%%%%%%%%%%%%%%%%%%%%%%%%%%%%%%%%%%%%%%%%%%%%%%%%%%%%%%%%%%%%%%%%%%%%%%
\section{Algorithm}

The algorithm that applies forward-backward algorithm and learning through EM is
described below \cite{ghahramani}.

\begin{figure}[!hbp]
\center{
\begin{tabbing}
{\tt LDS}\={\tt Inf}\={\tt erence}$(Y,A,C,Q,R,x,P)$\\
\> {\tt for } $t = 1$ {\tt to } $T$ \% forward recursion \\
\> \> $x_t^{t-1} = Ax_{t-1}^{t-1}$\\
\> \> $P_t^{t-1} = AP_{t-1}^{t-1}A' + Q$ {\tt if } $t>1$\\
\> \> $K_t = P_t^{t-1}C'(CP_t^{t-1}C'+R)^{-1}$\\
\> \> $x_t^t = x_t^{t-1} + K_t(y_t - Cx_t^{t-1})$\\
\> \> $P_t^t = P_t^{t-1} - K_tCP_t^{t-1}$\\
\> {\tt end}\\
\> {\tt initialize } $\hat{P}_{T,T-1}=(I-K_TC)AP_{T-1}^{T-1}$\\
\> \> {\tt for } $t=T$ {\tt to } $2$ \% backward recursion \\
\> \> $J_{t-1} = P_{t-1}^{t-1}A'(P_t^{t-1})^{-1}$\\
\> \> $\hat{x}_{t-1} = x_{t-1}^{t-1} + J_{t-1}(\hat{x}_{t} - Ax_{t-1}^{t-1})$\\
\> \> $\hat{P}_{t-1} = P_{t-1}^{t-1} + J_{t-1}(\hat{P}_t - P_t^{t-1})J_{t-1}'$\\
\> \> $\hat{P}_{t,t-1} = P_t^t J_{t-1}' + J_t(\hat{P}_{t+1,t} - AV_t^t)J_{t-1}'$
   {\tt if } $t<T$\\
\> {\tt end}\\
\> {\tt return }$\hat{x}_t, \hat{P}_t, \hat{P}_{t,t-1}$ {\tt for all } $t$
\end{tabbing}
}
\caption{LDS Inference Algorithm}
\vspace{0.6cm}
\end{figure}

\begin{figure}[!hbp]
\center{
\begin{tabbing}
{\tt LDS}\={\tt Learn}\=$(Y,k,$\=$\epsilon)$\\ \> {\tt initialize } $A, C, Q, R,
x_1^0, P_1^0$\\ \> $\alpha = \sum_t y_ty_t'$\\ \> {\tt while change in
  likelihood >} $\epsilon$\\ \> \> {\tt LDSInference}$(Y,A,C,Q,R,x_1^0,P_1^0)$
\% E-step\\ \> \> {\tt initialize } $\delta = 0, \gamma = 0, \sigma = 0$\\ \> \>
   {\tt for } $t = 1$ {\tt to } $T$\\ \> \> \> $\delta = \delta +
   y_t\hat{x}_t$\\ \> \> \> $\gamma = \gamma + \hat{x}_t\hat{x}_t' +
   \hat{P}_t$\\ \> \> \> $\beta = \beta + \hat{x}_t\hat{x}_{t-1}' + P_{t,t-1}$
       {\tt if } $t>1$\\ \> \> {\tt end}\\ \> \> $\gamma_1 = \gamma -
       \hat{x}_T\hat{x}_T' - \hat{P}_T$\\ \> \> $\gamma_2 = \gamma -
       \hat{x}_1\hat{x}_1' - \hat{P}_1$\\ \> \> \% M-step\\ \> \> $C = \delta
       \gamma^{-1}$\\ \> \> $R = (\gamma - C\delta')/T$\\ \> \> $A =
       \beta\gamma_1^{-1}$\\ \> \> $Q = (\gamma_2 - A\beta') / (T-1)$\\ \> \>
       $x_1^0 = \hat{x}_1$\\ \> \> $P_1^0 = \hat{P}_1$\\ \> {\tt end}\\ \> {\tt
         return } $A, C, Q, R, x_1^0, P_1^0$
\end{tabbing}
}
\caption{LDS Learning Algorithm}
\vspace{0.6cm}
\end{figure}

%%%%%%%%%%%%%%%%%%%%%%%%%%%%%%%%%%%%%%%%%%%%%%%%%%%%%%%%%%%%%%%%%%%%%%%%%%%

%%%%%%%%%%%%%%%%%%%%%%%%%%%%%%%%%%%%%%%%%%%%%%%%%%%%%%%%%%%%%%%%%%%%%%%%%%%%%
\section{Prediction with Kalman Filters}

Generating future values with a KF model is somewhat easier than generating
random values from an HMM. We know state space model assigns one hidden Gaussian
distribution per data point, which itself is modeled by another Gaussian. In
comparison, the number of hidden states of an HMM is specified by the modeler
and is usually less than the number of data points used for training.

The fact that we have as many hidden/observed random variables as data points
makes the following more intuitive: After the training phase of the KF which
includes forward recursion and smoothing, we have estimates for the mean and
variance for all Gaussian distributions. Another knowledge we have at hand is
the basic time transition equation for KFs - if we have an estimate at $t$ for
hidden state, we know how to propagate those estimates to the future at
$t+1$. Therefore, once training is complete, we can simply fast forward to the
emission Gaussian at time $T$, and using propagation equation generate a new
observation Gaussian at time $T+1$ which we use to generate a random number that
represents ``generatively'' an output at time $T+1$.

One point we have to be careful with is propagating the emission Gaussian, not
simply the hidden Gaussian. In the literature the basic time transition equation
is for the latter not the former, hence we need to expand the equation for $y_t$
by taking time transition into account:

Given,
\begin{eqnarray}
x_{t+1} = Ax_t + v_t \\
y_t = Cx_t + w_t
\end{eqnarray}
where $v_0,v_1,...$ is white Gaussian noise with $\mathbf{E}v_t = 0$ and $cov \:
v_t = Q$ and $w_0,w_1,...$ are also white Gaussian noise with $\mathbf{E}w_t =
0$ and $cov \: w_t = R$, we can determine $y_{t+1}$ as
\begin{eqnarray}
y_{t+1} &=& Cx_{t+1} + w_{t+1}\\
&=& C(Ax_t + v_t) + w_{t+1} \\
&=& CAx_t + Cv_t + w_{t+1}
\end{eqnarray}
which gives us the distribution of the emission Gaussian $y_{t+1}$. The
conditional expectation and variance of this distribution given $y_{1,..,t}$ is
\begin{eqnarray}
y_{t+1} &=& Cx_{t+1} + w_{t+1}\\
&=& C(Ax_t + v_t) + w_{t+1} \\
&=& CAx_t + Cv_t + w_{t+1} \\
\hat{y}_{t+1|t} &=& CA\hat{x}_{t|t} \\
cov(y_{t+1}|y_{0,...,t}) &=& CA P_{t|t}A'C' + CQC' + R
\end{eqnarray}
We can apply this recursive formula to have our KF generate $M$ many random
observations starting at $T$ up to $T+N$. The algorithm is

\begin{figure}[!hbp]
\center{
\begin{tabbing}
{\tt ob}\={\tt ser}\={\tt vation}\={\tt Genera}\={\tt tion}$(A,C,R,Q,\mu,\Sigma,T,N)$\\
\> $mu_x = mu[T-1];$\\
\> $sigma_x = sigma[T-1];$\\
\> {\tt for} $i = 1$ {\tt to } $N$\\
\> \> $mu_y = C A mu_x;$\\
\> \> $sigma_y = CA sigma_x A'C' + CQC' + R;$\\
\> \> $o[i] = $ {\tt gaussianNumberGeneration}$(mu_y, sigma_y);$\\
\> \> $mu_x = A mu_x$\\
\> \> $sigma_x = A  sigma_x A + Q$\\
\> {\tt end}\\
\> {\tt return} $o;$
\end{tabbing}
}
\caption{Generating Observations from an LDS Algorithm}
\vspace{0.6cm}
\end{figure}

